\documentclass{article} % For LaTeX2e
\usepackage{nips13submit_e,times}
\usepackage{hyperref}
\usepackage{url}
\usepackage{tabularx}
%\documentstyle[nips13submit_09,times,art10]{article} % For LaTeX 2.09


\title{Detecting ELL Writing}


\author{
Shayne Miel \\
SUNet ID: smiel\\
\texttt{smiel@stanford.edu} \\
}

% The \author macro works with any number of authors. There are two commands
% used to separate the names and addresses of multiple authors: \And and \AND.
%
% Using \And between authors leaves it to \LaTeX{} to determine where to break
% the lines. Using \AND forces a linebreak at that point. So, if \LaTeX{}
% puts 3 of 4 authors names on the first line, and the last on the second
% line, try using \AND instead of \And before the third author name.

\newcommand{\fix}{\marginpar{FIX}}
\newcommand{\new}{\marginpar{NEW}}

\nipsfinalcopy % Uncomment for camera-ready version

\begin{document}


\maketitle

\begin{abstract}
Determining whether an author is writing in their native language (L1) or a
second language (L2) is a problem that lies at the intersection of three
traditional NLP tasks: native language identification, similar language
identification, and detecting translationese.
In general, the goal of the language
learner is to improve their proficiency until their writing is indistinguishable
from that of a native speaker. By being able to automatically and reliably
determine whether a section of text looks like L1 or L2 text, areas of writing
that still need improvement can be brought to the learner's attention.
Additionally, the state of the art for correcting grammatical errors involves
using machine translation to translate from errorful text to corrected text\cite{chollampatt}
and there is interesting work being done in generating new training examples
by using machine translation to go
from error-free text to errorful text.\cite{liu} Both approaches could be enhanced by a
system that can tell how close the translation is to an L1 or L2 target.

I present a supervised deep learning method for determining the probability
that an essay was written by an English Language Learner (ELL) or a
native English writer, using document-level labels while presenting only sliding
windows of text to the model. A bidirectional LSTM acts as a convolution
filter for the text and a pooling layer provides the final document-level
prediction. My model improves upon a simple baseline by 0.2 AUC, but is
unfortunately still picking up on confounds due to the data collection process.
\end{abstract}

\section{Introduction}

One of the largest challenges for non-native English students is learning to
write in a way that reads like native English writing. Why blah

Examples

An automated means of providing feedback on areas of text that appear less
like native English writing would be useful to students learning the language.

This paper introduces a relatively simple model blah

Section 2.1 provides examples of similar work wrt the task and section 2.2 wrt
the approach. Section 3 describes the network in detail as well as several
modifications that were tested. Section 4 gives the particulars of the
experiment, including how the corpus was split into train/dev/test splits, the
metric used, and the hyperparamters tested. Finally, section 5 analyzes the
results and provides some next steps for future research.

\section{Related work}

\subsection{Task}

The task of ELL writing detection has not been studied extensively. Tomokiyo, et al.
attempt this task using the TODO corpus and a TODO. \cite{tomokiyo}
However, their work focuses
mostly on text transcribed from spoken recordings, rather than writing generated
by the ELL writer.

A similar task, native language identification, has received more attention.
Malmasi, et al, established the ETS TOEFL-11 corpus, which has been used by a
number of researchers. \cite{malmasi} Unfortunately for the purposes in this
paper, native English writers don't tend to take the TOEFL and as such are not
represented in this work. \cite{ionescu}

Classifying writing from similar language pairs (for instance TODO and TODO) is
also related to this task, in that the same words are being used but the grammatical
constructs are generally not followed in the same way. Gouette explores this
task using a TODO . \cite{goutte}

Finally, detecting translationese is also quite similar to this work in that
beginning ELL students tend to conceive the structure and content of their essays
in their primary language and then translate it to English when setting it
down in writing. Baroni has studied this problem with TODO \cite{baroni}

\subsection{Model}

One of the existing challenges for using RNNs and semantic vectors in general
is figuring out how best to encode long documents. The approach taken in this
paper is to use an LSTM as a convolution layer over sliding windows of text.
This approach bears some similarities to prior work. Tang,
et al. work with both a CNN and an LSTM to compose word vectors into sentence
vectors, and then compose the sentence vectors into document vectors using a
gated recurrent network.\cite{tang} Yang, et al. use a similar
word-layer/sentence-layer architecture, but also provide an attention mechanism
at each layer to enhance the quality of the output vectors.\cite{yang} The model in this
paper differs from both of these in that the second level vectors contain
overlapping information, due to the overlapping nature of sliding windows.

In \cite{mikolov}, Mikolov describes a way to generate
document vectors by adding a document token to the context of every word in the
document in what is otherwise a standard word2vec model. While this approach
is better than simply averaging all of the word vectors in the document, it
loses important information about the order of tokens that are crucial to
detecting ungrammatical English.

Collobert, et al. use
a convolutional layer with a max pooling layer to generate window vectors
around each word in a part of speech tagging task.\cite{collobert} This is similar
to the model presented here, except that I use an LSTM \textit{as} the convolution.


\section{Approach}

\subsection{Data collection}

\begin{table}[t]
\caption{ELL and Native English Corpora}
\label{data-table}
\begin{center}
\begin{tabularx}{\textwidth}{l r r X}
\multicolumn{1}{c}{\bf CORPUS} &\multicolumn{1}{c}{\bf $n$ ESSAYS} &\multicolumn{1}{c}{\bf $n$ PROMPTS} &\multicolumn{1}{c}{\bf L1s}
\\ \hline \\
\hline
\href{http://language.sakura.ne.jp/icnale/}{ICNALE} & $5,600$ & $2$ & CHN, \textbf{ENG}, FIL, HKG, IND, JPN, \newline KOR, PAK, SIN, THA, TWN \\
\href{http://www.comp.nus.edu.sg/~nlp/conll14st.html\#nucle32}{NUCLE} & $1,397$ & $3$ & CHN \\
\href{http://ilexir.co.uk/datasets/index.html}{FCE} & $2,481$ & $44$ & CAT, CHN, FRA, GER, GRC, ITA, \newline JPN, KOR, NL, POL, PRT, RUS, \newline SPA, SWE, THA, TUR \\
\href{https://meta-toolkit.org/data/2016-01-26/ceeaus.tar.gz}{CEEAUS} & $1,008$ & $2$ & CHN, \textbf{ENG}, JPN \\
\href{http://www.u-sacred-heart.ac.jp/okugiri/links/moecs/links/data/data.html}{MOECS} & $199$ & $1$ & \textbf{ENG}, JPN \\
\href{http://koreanlearnercorpusblog.blogspot.be/p/corpus.html}{Gachon} & $15,831$ & $20$ & KOR \\
\href{http://www.bfsu-corpus.org/static/corpora/TECCL_Corpus_V1.1.zip}{TECCL} & $9,864$ & $???$ & CHN \\
Private A & $550$ & $1$ & KOR \\
Private B & $4,694$ & $4$ & CHN \\
\href{https://catalog.ldc.upenn.edu/LDC2014T06}{TOEFL-11} & $12,100$ & $8$ & ARA, CHN, FRA, GER, IND, ITA, \newline JPN, KOR, SPA, TEL, TUR \\
Private C & $41,227$ & $53$ & \textbf{ENG} \\
Private D & $29,559$ & $21$ & \textbf{ENG} \\
\href{https://www.kaggle.com/c/asap-aes/data}{ASAP} & $17,677$ & $8$ & \textbf{ENG} \\
\end{tabularx}
\end{center}
\end{table}

\begin{table}
\caption{L1s}
\label{L1s-table}
\begin{center}
\begin{tabularx}{\textwidth}{l l l l l l }
\multicolumn{1}{c}{\bf ABBREVIATION} & \multicolumn{1}{c}{\bf LANGUAGE} \\
\hline
ARA & Arabic &    GRC  & Greek   &          PRT  & Portuguese \\
BUL & Bulgarian &      HKG  &  Hong Kong Cantonese       &    RUS  & Russian \\
CAT & Catalan &    IND  & Indian languages   &           SIN  & Singapore languages \\
CHN & Chinese &    ITA  & Italian        &   SPA  & Spanish \\
CZE & Czech &     JPN  & Japanese         &     SWE  & Swedish \\
ENG & English &    KOR  & Korean         &   TEL  & Telugu \\
FIL & Filipino &       NL  &  Dutch      &       THA &  Thai \\
FIN & Finnish &    NOR &  Norwegian       &      TSW  & Tswana \\
FRA & French &     PAK  & Urdu         &     TUR  & Turkish \\
GER & German &     POL  & Polish    &        TWN  & Taiwanese
\end{tabularx}
\end{center}
\end{table}

One of the reasons that there are not many papers applying deep learning to ELL
related problems is the lack of a sufficiently sized corpus. In order to collect
enough data to make use of a deep neural network, I had to stitch together a
number of freely available research corpora, as well as some privately held
corpora provided by \href{http://turnitin.com/}{Turnitin (turnitin.com)}. To
protect the privacy of the students, I will refer to the data sets from Turnitin
as Private A, Private B, etc. They are all essays written online by students in
6th-12th grade, as well as early college. Details of all 14 corpora are described
in Table~\ref{data-table}. The L1 abbreviations are listed in Table~\ref{L1s-table}.

\subsection{Data cleaning}

\begin {table}
\caption {Confounds}
\label{table:confounds}
\begin{center}
%\newcolumntype{L}{>{\raggedright\arraybackslash}X}
\begin{tabularx}{\textwidth}{X | X}
Potential Confound & Mitigation \\
\hline
Formatting differences may cause the model to learn that an entire sub-dataset
is native or ELL. &
\begin{enumerate}
    \item Use entirely separate data sets for train and test splits.
    \item Restrict the formatting so that there is no repeated whitespace and
    all characters fall between \textbackslash x32 (Space) and \textbackslash x126
    (\texttildelow).
\end{enumerate} \\
\hline
Prompt differences may cause the model to learn that an entire prompt is native
or ELL. & \begin{enumerate}
    \item Use entirely separate prompts for train, dev and test splits.
    \item Make sure the prompts we do have that contain both ELL and native writing
    are present in the test set.
    \item Control for the contribution of the prompt in the model (add an explanatory
    variable if using regression, add a secondary objective ala \cite{zhong}
    if using a neural network, etc).
\end{enumerate} \\
\hline
Native writers tend to write longer essays and longer sentences than ELL students.
While ``write more'' is something we would want to encourage, it doesn't provide the
kind of meaningful proficiency feedback we'd want to give. & Rather than using
the full essay or even a particular sentence as an input, I will use sliding
windows of $n=100$ characters and take either the average or the max of the predicted
probabilities.
\end{tabularx}
\end{center}
\end{table}

Collecting data from such a varied set of corpora also presents its own challenges,
however. There are many potential differences between data sets collected under
different conditions and in different years/locations. In order to prevent the
model from learning confounds that are unrelated to the proficiency of the writer,
I cleaned the data as much as I was able to. Potential confounds and steps taken
to mitigate the risk are presented in Table~\ref{table:confounds}.

\subsection{Neural network description}

\subsection{Modifications}


\section{Experiments}

train, dev, test split
metric
baseline
abbreviations
hyperparameters

dev/test table

one essay per model

\section{Conclusion}

\subsection{Analysis}
Neural net is able to beat the simple baseline provided, but it does so by
picking up on confounds in the data that are unrelated to the desired output
of how similar is this section of text native writing


\subsection{Future steps}
Language model trained on large corpus of english

Collecting large corpus of native/non-native writing on the same topic

Collecting proficiency scores on the writing, rather than using L1 as
a proxy.

\subsection{Figures}

All artwork must be neat, clean, and legible. Lines should be dark
enough for purposes of reproduction; art work should not be
hand-drawn. The figure number and caption always appear after the
figure. Place one line space before the figure caption, and one line
space after the figure. The figure caption is lower case (except for
first word and proper nouns); figures are numbered consecutively.

Make sure the figure caption does not get separated from the figure.
Leave sufficient space to avoid splitting the figure and figure caption.

You may use color figures.
However, it is best for the
figure captions and the paper body to make sense if the paper is printed
either in black/white or in color.
\begin{figure}[h]
\begin{center}
%\framebox[4.0in]{$\;$}
\fbox{\rule[-.5cm]{0cm}{4cm} \rule[-.5cm]{4cm}{0cm}}
\end{center}
\caption{Sample figure caption.}
\end{figure}

\subsection{Tables}

All tables must be centered, neat, clean and legible. Do not use hand-drawn
tables. The table number and title always appear before the table. See
Table~\ref{sample-table}.

Place one line space before the table title, one line space after the table
title, and one line space after the table. The table title must be lower case
(except for first word and proper nouns); tables are numbered consecutively.

\begin{table}[t]
\caption{Sample table title}
\label{sample-table}
\begin{center}
\begin{tabular}{ll}
\multicolumn{1}{c}{\bf PART}  &\multicolumn{1}{c}{\bf DESCRIPTION}
\\ \hline \\
Dendrite         &Input terminal \\
Axon             &Output terminal \\
Soma             &Cell body (contains cell nucleus) \\
\end{tabular}
\end{center}
\end{table}

\section{Final instructions}
Do not change any aspects of the formatting parameters in the style files.
In particular, do not modify the width or length of the rectangle the text
should fit into, and do not change font sizes (except perhaps in the
\textbf{References} section; see below). Please note that pages should be
numbered.

\section{Preparing PostScript or PDF files}

Please prepare PostScript or PDF files with paper size ``US Letter'', and
not, for example, ``A4''. The -t
letter option on dvips will produce US Letter files.

Fonts were the main cause of problems in the past years. Your PDF file must
only contain Type 1 or Embedded TrueType fonts. Here are a few instructions
to achieve this.

\begin{itemize}

\item You can check which fonts a PDF files uses.  In Acrobat Reader,
select the menu Files$>$Document Properties$>$Fonts and select Show All Fonts. You can
also use the program \verb+pdffonts+ which comes with \verb+xpdf+ and is
available out-of-the-box on most Linux machines.

\item The IEEE has recommendations for generating PDF files whose fonts
are also acceptable for NIPS. Please see
\url{http://www.emfield.org/icuwb2010/downloads/IEEE-PDF-SpecV32.pdf}

\item LaTeX users:

\begin{itemize}

\item Consider directly generating PDF files using \verb+pdflatex+
(especially if you are a MiKTeX user).
PDF figures must be substituted for EPS figures, however.

\item Otherwise, please generate your PostScript and PDF files with the following commands:
\begin{verbatim}
dvips mypaper.dvi -t letter -Ppdf -G0 -o mypaper.ps
ps2pdf mypaper.ps mypaper.pdf
\end{verbatim}

Check that the PDF files only contains Type 1 fonts.
%For the final version, please send us both the Postscript file and
%the PDF file.

\item xfig "patterned" shapes are implemented with
bitmap fonts.  Use "solid" shapes instead.
\item The \verb+\bbold+ package almost always uses bitmap
fonts.  You can try the equivalent AMS Fonts with command
\begin{verbatim}
\usepackage[psamsfonts]{amssymb}
\end{verbatim}
 or use the following workaround for reals, natural and complex:
\begin{verbatim}
\newcommand{\RR}{I\!\!R} %real numbers
\newcommand{\Nat}{I\!\!N} %natural numbers
\newcommand{\CC}{I\!\!\!\!C} %complex numbers
\end{verbatim}

\item Sometimes the problematic fonts are used in figures
included in LaTeX files. The ghostscript program \verb+eps2eps+ is the simplest
way to clean such figures. For black and white figures, slightly better
results can be achieved with program \verb+potrace+.
\end{itemize}
\item MSWord and Windows users (via PDF file):
\begin{itemize}
\item Install the Microsoft Save as PDF Office 2007 Add-in from
\url{http://www.microsoft.com/downloads/details.aspx?displaylang=en\&familyid=4d951911-3e7e-4ae6-b059-a2e79ed87041}
\item Select ``Save or Publish to PDF'' from the Office or File menu
\end{itemize}
\item MSWord and Mac OS X users (via PDF file):
\begin{itemize}
\item From the print menu, click the PDF drop-down box, and select ``Save
as PDF...''
\end{itemize}
\item MSWord and Windows users (via PS file):
\begin{itemize}
\item To create a new printer
on your computer, install the AdobePS printer driver and the Adobe Distiller PPD file from
\url{http://www.adobe.com/support/downloads/detail.jsp?ftpID=204} {\it Note:} You must reboot your PC after installing the
AdobePS driver for it to take effect.
\item To produce the ps file, select ``Print'' from the MS app, choose
the installed AdobePS printer, click on ``Properties'', click on ``Advanced.''
\item Set ``TrueType Font'' to be ``Download as Softfont''
\item Open the ``PostScript Options'' folder
\item Select ``PostScript Output Option'' to be ``Optimize for Portability''
\item Select ``TrueType Font Download Option'' to be ``Outline''
\item Select ``Send PostScript Error Handler'' to be ``No''
\item Click ``OK'' three times, print your file.
\item Now, use Adobe Acrobat Distiller or ps2pdf to create a PDF file from
the PS file. In Acrobat, check the option ``Embed all fonts'' if
applicable.
\end{itemize}

\end{itemize}
If your file contains Type 3 fonts or non embedded TrueType fonts, we will
ask you to fix it.

\subsection{Margins in LaTeX}

Most of the margin problems come from figures positioned by hand using
\verb+\special+ or other commands. We suggest using the command
\verb+\includegraphics+
from the graphicx package. Always specify the figure width as a multiple of
the line width as in the example below using .eps graphics
\begin{verbatim}
   \usepackage[dvips]{graphicx} ...
   \includegraphics[width=0.8\linewidth]{myfile.eps}
\end{verbatim}
or % Apr 2009 addition
\begin{verbatim}
   \usepackage[pdftex]{graphicx} ...
   \includegraphics[width=0.8\linewidth]{myfile.pdf}
\end{verbatim}
for .pdf graphics.
See section 4.4 in the graphics bundle documentation (\url{http://www.ctan.org/tex-archive/macros/latex/required/graphics/grfguide.ps})

A number of width problems arise when LaTeX cannot properly hyphenate a
line. Please give LaTeX hyphenation hints using the \verb+\-+ command.


\subsubsection*{Acknowledgments}

Use unnumbered third level headings for the acknowledgments. All
acknowledgments go at the end of the paper. Do not include
acknowledgments in the anonymized submission, only in the
final paper.

\begin{thebibliography}{6}

\bibitem{baroni}
    Baroni, Marco, and Silvia Bernardini.
    "A new approach to the study of translationese: Machine-learning the difference
    between original and translated text."
    \emph{Literary and Linguistic Computing} 21.3 (2006): 259-274.

\bibitem{chollampatt}
    Chollampatt, Shamil, Kaveh Taghipour, and Hwee Tou Ng.
    "Neural network translation models for grammatical error correction."
    \emph{arXiv preprint} arXiv:1606.00189 (2016).

\bibitem{goutte}
    Goutte, Cyril, et al.
    "Discriminating similar languages: Evaluations and explorations."
    \emph{arXiv preprint} arXiv:1610.00031 (2016).

\bibitem{liu}
    Liu, Zhuoran, and Yang Liu.
    "Exploiting Unlabeled Data for Neural Grammatical Error Detection."
    \emph{arXiv preprint} arXiv:1611.08987 (2016).

\bibitem{malmasi}
    Malmasi, Shervin, Joel Tetreault, and Mark Dras.
    "Oracle and human baselines for native language identification."
    \emph{Proceedings of the Tenth Workshop on Innovative Use of NLP for Building
    Educational Applications.} 2015.

\bibitem{tomoyiko}
    Tomokiyo, Laura Mayfield, and Rosie Jones.
    "You're not from 'round here, are you?: naive Bayes detection of non-native utterance text."
    {Proceedings of the second meeting of the North American Chapter of the Association
    for Computational Linguistics on Language technologies.} Association for Computational Linguistics, 2001.

\end{thebibliography}

\end{document}
